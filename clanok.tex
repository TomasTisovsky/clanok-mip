% Metódy inžinierskej práce

\documentclass[10pt,twoside,slovak,a4paper]{article}

\usepackage[slovak]{babel}
%\usepackage[T1]{fontenc}
\usepackage[IL2]{fontenc} % lepšia sadzba písmena Ľ než v T1
\usepackage[utf8]{inputenc}
\usepackage{graphicx}
\usepackage{url} % príkaz \url na formátovanie URL
\usepackage{hyperref} % odkazy v texte budú aktívne (pri niektorých triedach dokumentov spôsobuje posun textu)

\usepackage{cite}
%\usepackage{times}

\pagestyle{headings}

\title{Životný cyklus dátových skladov\thanks{Semestrálny projekt v predmete Metódy inžinierskej práce, ak. rok 2021/22, vedenie: Ing. Vladimír Mlynarovič, PhD.}} % meno a priezvisko vyučujúceho na cvičeniach

\author{Tomáš Tisovský\\[2pt]
	{\small Slovenská technická univerzita v Bratislave}\\
	{\small Fakulta informatiky a informačných technológií}\\
	{\small \texttt{xtisovskyt@stuba.sk}}
	}

\date{\small 30. september 2015} % upravte



\begin{document}

\maketitle

\begin{abstract}
Témou tohto článku je proces tvorby dátových skladov vrámci oblasti Business Intelligence. Vzhľadom na pomerne vysokú komplexnosť DW/BI systémov je využitie agilných praktík v tejto oblasti nutnosťou pre efektívnu tvorbu daných projektov. Pri znalosti metód a postupov pri vývoji dátových skladov je práca na projekte organizovana a môže dôjsť k výraznému ušetreniu času, či finančných prostriedkov. Článok má za cieľ predstaviť najlepšie techniky, praktiky a prístupy pre tento obor. Článok sa venuje rôznym etapám životného cyklu dátoveho skladu ako plánovanie a riadenie projektu, definícia a zber požiadaviek, technologická fáza, dátová fáza, aplikačná fáza a údržba.
Tieto ciele článok spĺňa opísaním kompletného životného cyklu dátových skladov. 
\end{abstract}



\section{Úvod}

Tento článok sa venuje životnému cyklu dátových skladov. Technológia dátových skladov predstavuje v súčasnosti jeden z najvýznamnejších trendov v rozvoji podnikových informačných systémov. Dátový sklad (Data Warehouse) možno definovať mnohými spôsobmi. Za základ však budeme považovať definície jedného zo zakladateľov DWH, Williama Inmon.„Dátový sklad je integrovaný, subjektovo orientovaný, stály a časovo rozlíšený súhrn dát, usporiadaný pre podporu potrieb manažmentu“\cite{Novotny2005} 

Životný cyklus prebieha v niekolkých etapách. Poznanie etáp projektu je dôležité pre všetkých účastníkov projektu, teda manažérov, analytikov, návrhárov, či vývojárov na vykonanie správnych úloh v správny čas. Pri vytvorení takéhoto softvéru sa kladie hlavný dôraz na požiadavky užívateľov, iteratívnosť a dimenzionalitu v poňatí štruktúrovaných dát. Za štandard v tejto oblasti sa považuje životný cyklus od  Ralpha Kimballa. Tento model využíva agilný prístup pre jeho vyššiu efektivitu a úspešnosť. Tento model je zobrazený v Kimball Lifecycle diagram (obr. nižšie). Tento diagram poskytuje celkový plán znázornujúci postupnosť úloh na vysokej úrovni potrebnných  pre úspešné DW/BI projekty. 

Model podľa Kimballa začína plánovaním. V tejto fáze je potrebné určiť rozsah projektu a potrebné zdroje. V tejto fáze sa začínajú aj riadiace povinnosti, ktoré pretrvávajú počas celého zvyšku projektu.  
Na fázu plánovania nadväzuje fáza definovania uživateľských požiadaviek. Táto fáza je pre projekt klúčová pretože ovplyvnuje všetky nasledovné fázy projektu. Medzi fázou plánovania a fázou zberu požiadaviek je potreba úzkej vzájomnej spolupráce.
Horná časť diagramu sa venuje technologickej stránke projektu kde prebieha technologická architektúra. Po dokončení architektúri sa vyberú vhodné nástroje na tvorbu sofvéru. 
Stredná časť diagramu sa zaoberá dátovej stránke projektu. V tejto fáze sa vývojári venujú dátam, operáciam s nimi.  Vzniká tu  multidimenzionálny model, ktorý je podstatou DW/BI projekov. Taktiež sa tu tvorí fyzický model a prebehne ETL proces(extract, transform, load). 
Spodná časť diagramu sa sustredí na výstupy pre uživateľov vo forme multidimenzionalnej aplikácie. 
V sekcii s názvom rast sú vyjadrené praktiky inkrementálnosti. Táto sekcia  hovorí o tom , že pri každom prírastku dát by sa vývojári mali vracať k plánu a držať sa požiadaviek uživateľov. Nasledujúca časť článku je venovaná jednotlivím etapám životného cyklu. \cite{Kimball2013}






\section{Plánovanie a riadenie projektu} \label{nejaka}

Dlhodobý cieľ projektu dátového skladu počíta nielen s jeho vybudovaním, ale definuje aj stratégiu správy dátového skladu, pričom počíta s dokumentáciou dátového skladu a školením jeho užívateľov. V tejto fáze sú definované aj základy architektúry podnikového dátového skladu. Behom fázy definície sa definuje rozsah a cieľ prírastkového vývoja. Vytvorí sa počiatočný prírastok, konceptuálny model, zdokumentujú sa zdroje dát a presne sa vymedzí rozsah kvality týchto dát. Je navrhnutá ako aj architektúra dátového skladu, tak aj architektúra technických prostriedkov. V tejto fáze máme najlepšiu príležitosť zamerať sa na pochopenie štruktúry operačných a externých zdrojov dát. Stanovia sa krátkodobé a dlhodobé obchodné ciele, pre podporu ktorých je dátový sklad budovaný.\cite{Arnost2007}

Proces riadenia projektov sa týka koordinácie ľudských, finančných a materiálnych zdrojov, je zameraný na dosiahnutie dopredu stanovených cieľov v danom rozsahu, čase, nákladoch, kvalite a spokojnosti účastníkov projektu. \cite{Tvrdikova2008}

Tak ako pri každom projekte aj pri DW/BI projekte je kľúčový vedúci projektu. Pre projekt je tiež dôležitá úloha biznis analytika, ktorý by mal mať dobré povedomie o spolupráci IT a biznisu. Úlohou dátového analytika je analyzovať kvalitu dát, ich kvalitu a granularitu. 
Úlohou externého konzultanta je obyčajne transfer poznatkov podniku, vyškolenie a riadenie ľudí, vedenie projektov. Konzultačné služby v oblasti dátových skladov sa spravidla týkajú výberu hardvéru a softvéru, návrhu architektúry a optimálnej zostavy softvérových technológií a zaistenie využitia najnovších informačných technológií v rôznych oblastiach podnikania. Konzultanti poskytujú služby tiež v oblasti systémovej integrácie, konvergencie služieb a technológií, služby v oblasti on-line bezpečnosti, vytváranie webových stránok, podnikové informačné systémy. IT konzultant všeobecne pomáha firmám pochopiť, akým spôsobom môžu využiť technológie pre svoj prospech. \cite{Arnost2007}

Pre projekt je kľúčové vymedzenie nákladov:

\begin{itemize}
\item Hardvér - náklady na hardvér nie sú zanedbateľné, ale z hľadiska filozofie dátového skladu sa jedná o technické prostriedky, ktoré sú nahraditeľné. Jedná sa o technické prostriedky, nie o dáta ktoré sú v nich uložené. Pre rýchly prístup k obrovskému množstvu dát je potrebné mať výkonné servery, alebo dátové centrá.\cite{Novotny2005}
\item Softvér - Nástroje pre vytváranie dátových skladov a analýzu dát sú veľmi drahou záležitosťou. Čím ďalej viac sa presadzuje trend integrácie analytických služieb priamo do inštalácií databázových serverov. Poplatok za analytické služby je buď zahrnutý priamo v cene databázového serveru, alebo sa licenčné poplatky platia zvlášť.\cite{Lacko2003} 

\end{itemize}




Taktiež je dôležité aj určenie benefitov, ktoré daný systém prinesie, teda určiť ktoré konkrétne rozhodovacie procesy v spoločnosti budú podporené.






\section{Iná časť} \label{ina}

Základným problémom je teda\ldots{} Najprv sa pozrieme na nejaké vysvetlenie (časť~\ref{ina:nejake}), a potom na ešte nejaké (časť~\ref{ina:nejake}).\footnote{Niekedy môžete potrebovať aj poznámku pod čiarou.}

Môže sa zdať, že problém vlastne nejestvuje\cite{Coplien:MPD}, ale bolo dokázané, že to tak nie je~\cite{Czarnecki:Staged, Czarnecki:Progress}. Napriek tomu, aj dnes na webe narazíme na všelijaké pochybné názory\cite{PLP-Framework}. Dôležité veci možno \emph{zdôrazniť kurzívou}.


\subsection{Nejaké vysvetlenie} \label{ina:nejake}

Niekedy treba uviesť zoznam:

\begin{itemize}
\item jedna vec
\item druhá vec
	\begin{itemize}
	\item x
	\item y
	\end{itemize}
\end{itemize}

Ten istý zoznam, len číslovaný:

\begin{enumerate}
\item jedna vec
\item druhá vec
	\begin{enumerate}
	\item x
	\item y
	\end{enumerate}
\end{enumerate}


\subsection{Ešte nejaké vysvetlenie} \label{ina:este}

\paragraph{Veľmi dôležitá poznámka.}
Niekedy je potrebné nadpisom označiť odsek. Text pokračuje hneď za nadpisom.



\section{Dôležitá časť} \label{dolezita}




\section{Ešte dôležitejšia časť} \label{dolezitejsia}




\section{Záver} \label{zaver} % prípadne iný variant názvu



%\acknowledgement{Ak niekomu chcete poďakovať\ldots}


% týmto sa generuje zoznam literatúry z obsahu súboru literatura.bib podľa toho, na čo sa v článku odkazujete
\bibliography{literatura}
\bibliographystyle{plain} % prípadne alpha, abbrv alebo hociktorý iný
\end{document}
