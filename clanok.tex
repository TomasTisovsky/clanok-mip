% Metódy inžinierskej práce

\documentclass[10pt,twoside,slovak,a4paper]{article}

\usepackage[slovak]{babel}
%\usepackage[T1]{fontenc}
\usepackage[IL2]{fontenc} % lepšia sadzba písmena Ľ než v T1
\usepackage[utf8]{inputenc}
\usepackage{graphicx}
\usepackage{url} % príkaz \url na formátovanie URL
\usepackage{hyperref} % odkazy v texte budú aktívne (pri niektorých triedach dokumentov spôsobuje posun textu)

\usepackage{cite}
%\usepackage{times}

\pagestyle{headings}

\title{Životný cyklus dátových skladov\thanks{Semestrálny projekt v predmete Metódy inžinierskej práce, ak. rok 2021/22, vedenie: Ing. Vladimír Mlynarovič, PhD.}} % meno a priezvisko vyučujúceho na cvičeniach

\author{Tomáš Tisovský\\[2pt]
	{\small Slovenská technická univerzita v Bratislave}\\
	{\small Fakulta informatiky a informačných technológií}\\
	{\small \texttt{xtisovskyt@stuba.sk}}
	}

\date{\small 30. september 2015} % upravte



\begin{document}

\maketitle

\begin{abstract}
Témou tohto článku je proces tvorby dátových skladov vrámci oblasti Business Intelligence. Vzhľadom na pomerne vysokú komplexnosť DW/BI systémov je využitie agilných praktík v tejto oblasti nutnosťou pre efektívnu tvorbu daných projektov. Pri znalosti metód a postupov pri vývoji dátových skladov je práca na projekte organizovana a môže dôjsť k výraznému ušetreniu času, či finančných prostriedkov. Článok má za cieľ predstaviť najlepšie techniky, praktiky a prístupy pre tento obor. Článok sa venuje rôznym etapám životného cyklu dátoveho skladu ako plánovanie a riadenie projektu, definícia a zber požiadaviek, technologická fáza, dátová fáza, aplikačná fáza a údržba.
Tieto ciele článok spĺňa opísaním kompletného životného cyklu dátových skladov. 
\end{abstract}



\section{Úvod}

Tento článok sa venuje životnému cyklu dátových skladov. Technológia dátových skladov predstavuje v súčasnosti jeden z najvýznamnejších trendov v rozvoji podnikových informačných systémov. Dátový sklad (Data Warehouse) možno definovať mnohými spôsobmi. Za základ však budeme považovať definície jedného zo zakladateľov DWH, Williama Inmon.„Dátový sklad je integrovaný, subjektovo orientovaný, stály a časovo rozlíšený súhrn dát, usporiadaný pre podporu potrieb manažmentu“\cite{Novotny2005} 

Životný cyklus prebieha v niekolkých etapách. Poznanie etáp projektu je dôležité pre všetkých účastníkov projektu, teda manažérov, analytikov, návrhárov, či vývojárov na vykonanie správnych úloh v správny čas. Pri vytvorení takéhoto softvéru sa kladie hlavný dôraz na požiadavky užívateľov, iteratívnosť a dimenzionalitu v poňatí štruktúrovaných dát. Za štandard v tejto oblasti sa považuje životný cyklus od  Ralpha Kimballa. Tento model využíva agilný prístup pre jeho vyššiu efektivitu a úspešnosť. Tento model je zobrazený v Kimball Lifecycle diagram (obr. nižšie). Tento diagram poskytuje celkový plán znázornujúci postupnosť úloh na vysokej úrovni potrebnných  pre úspešné DW/BI projekty. 

Model podľa Kimballa začína plánovaním. V tejto fáze je potrebné určiť rozsah projektu a potrebné zdroje. V tejto fáze sa začínajú aj riadiace povinnosti, ktoré pretrvávajú počas celého zvyšku projektu.  
Na fázu plánovania nadväzuje fáza definovania uživateľských požiadaviek. Táto fáza je pre projekt klúčová pretože ovplyvnuje všetky nasledovné fázy projektu. Medzi fázou plánovania a fázou zberu požiadaviek je potreba úzkej vzájomnej spolupráce.
Horná časť diagramu sa venuje technologickej stránke projektu kde prebieha technologická architektúra. Po dokončení architektúri sa vyberú vhodné nástroje na tvorbu sofvéru. 
Stredná časť diagramu sa zaoberá dátovej stránke projektu. V tejto fáze sa vývojári venujú dátam, operáciam s nimi.  Vzniká tu  multidimenzionálny model, ktorý je podstatou DW/BI projekov. Taktiež sa tu tvorí fyzický model a prebehne ETL proces(extract, transform, load). 
Spodná časť diagramu sa sustredí na výstupy pre uživateľov vo forme multidimenzionalnej aplikácie. 
V sekcii s názvom rast sú vyjadrené praktiky inkrementálnosti. Táto sekcia  hovorí o tom , že pri každom prírastku dát by sa vývojári mali vracať k plánu a držať sa požiadaviek uživateľov. Nasledujúca časť článku je venovaná jednotlivím etapám životného cyklu. \cite{Kimball2013}






\section{Nejaká časť} \label{nejaka}

Z obr.~\ref{f:rozhod} je všetko jasné. 

\begin{figure*}[tbh]
\centering
%\includegraphics[scale=1.0]{diagram.pdf}
Aj text môže byť prezentovaný ako obrázok. Stane sa z neho označný plávajúci objekt. Po vytvorení diagramu zrušte znak \texttt{\%} pred príkazom \verb|\includegraphics| označte tento riadok ako komentár (tiež pomocou znaku \texttt{\%}).
\caption{Rozhodujúci argument.}
\label{f:rozhod}
\end{figure*}



\section{Iná časť} \label{ina}

Základným problémom je teda\ldots{} Najprv sa pozrieme na nejaké vysvetlenie (časť~\ref{ina:nejake}), a potom na ešte nejaké (časť~\ref{ina:nejake}).\footnote{Niekedy môžete potrebovať aj poznámku pod čiarou.}

Môže sa zdať, že problém vlastne nejestvuje\cite{Coplien:MPD}, ale bolo dokázané, že to tak nie je~\cite{Czarnecki:Staged, Czarnecki:Progress}. Napriek tomu, aj dnes na webe narazíme na všelijaké pochybné názory\cite{PLP-Framework}. Dôležité veci možno \emph{zdôrazniť kurzívou}.


\subsection{Nejaké vysvetlenie} \label{ina:nejake}

Niekedy treba uviesť zoznam:

\begin{itemize}
\item jedna vec
\item druhá vec
	\begin{itemize}
	\item x
	\item y
	\end{itemize}
\end{itemize}

Ten istý zoznam, len číslovaný:

\begin{enumerate}
\item jedna vec
\item druhá vec
	\begin{enumerate}
	\item x
	\item y
	\end{enumerate}
\end{enumerate}


\subsection{Ešte nejaké vysvetlenie} \label{ina:este}

\paragraph{Veľmi dôležitá poznámka.}
Niekedy je potrebné nadpisom označiť odsek. Text pokračuje hneď za nadpisom.



\section{Dôležitá časť} \label{dolezita}




\section{Ešte dôležitejšia časť} \label{dolezitejsia}




\section{Záver} \label{zaver} % prípadne iný variant názvu



%\acknowledgement{Ak niekomu chcete poďakovať\ldots}


% týmto sa generuje zoznam literatúry z obsahu súboru literatura.bib podľa toho, na čo sa v článku odkazujete
\bibliography{literatura}
\bibliographystyle{plain} % prípadne alpha, abbrv alebo hociktorý iný
\end{document}
